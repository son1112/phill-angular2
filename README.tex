% Created 2016-05-23 Mon 22:01
\documentclass[11pt]{article}
\usepackage[utf8]{inputenc}
\usepackage[T1]{fontenc}
\usepackage{fixltx2e}
\usepackage{graphicx}
\usepackage{longtable}
\usepackage{float}
\usepackage{wrapfig}
\usepackage{rotating}
\usepackage[normalem]{ulem}
\usepackage{amsmath}
\usepackage{textcomp}
\usepackage{marvosym}
\usepackage{wasysym}
\usepackage{amssymb}
\usepackage{hyperref}
\tolerance=1000
\date{\today}
\title{README}
\hypersetup{
  pdfkeywords={},
  pdfsubject={},
  pdfcreator={Emacs 24.5.1 (Org mode 8.2.10)}}
\begin{document}

\maketitle
\tableofcontents

\textbf{Prairie Hill Learning Center}

NAV => \ref{sec-5} / \ref{sec-4} / \texttt{Index} / \ref{sec-3} / \ref{sec-2-4}

\section{Config}
\label{sec-1}

\subsection{: package.json}
\label{sec-1-1}
\begin{verbatim}
package.json
\end{verbatim}
\begin{verbatim}
{
  "name": "phill-angular2",
  "version": "1.0.0",
  "scripts": {
    "start": "tsc && concurrently \"npm run tsc:w\" \"npm run lite\" ",
    "lite": "lite-server",
    "postinstall": "typings install",
    "tsc": "tsc",
    "tsc:w": "tsc -w",
    "typings": "typings"
  },
  "license": "ISC",
  "dependencies": {
    "@angular/common":  "2.0.0-rc.1",
    "@angular/compiler":  "2.0.0-rc.1",
    "@angular/core":  "2.0.0-rc.1",
    "@angular/http":  "2.0.0-rc.1",
    "@angular/platform-browser":  "2.0.0-rc.1",
    "@angular/platform-browser-dynamic":  "2.0.0-rc.1",
    "@angular/router":  "2.0.0-rc.1",
    "@angular/router-deprecated":  "2.0.0-rc.1",
    "@angular/upgrade":  "2.0.0-rc.1",
    "systemjs": "0.19.27",
    "core-js": "^2.4.0",
    "reflect-metadata": "^0.1.3",
    "rxjs": "5.0.0-beta.6",
    "zone.js": "^0.6.12",
    "angular2-in-memory-web-api": "0.0.10",
    "bootstrap": "^3.3.6"
  },
  "devDependencies": {
    "concurrently": "^2.0.0",
    "lite-server": "^2.2.0",
    "typescript": "^1.8.10",
    "typings":"^0.8.1"
  }
}
\end{verbatim}

\subsection{: tsconfig.json}
\label{sec-1-2}
\begin{verbatim}
tsconfig.json
\end{verbatim}
\begin{verbatim}
{
  "compilerOptions": {
    "target": "es5",
    "module": "commonjs",
    "moduleResolution": "node",
    "sourceMap": true,
    "emitDecoratorMetadata": true,
    "experimentalDecorators": true,
    "removeComments": false,
    "noImplicitAny": false
  },
  "exclude": [
    "node_modules",
    "typings/main",
    "typings/main.d.ts"
  ]
}
\end{verbatim}

\subsection{: typings.json}
\label{sec-1-3}
\begin{verbatim}
typings.json
\end{verbatim}
\begin{verbatim}
{
  "ambientDependencies": {
    "core-js": "registry:dt/core-js#0.0.0+20160317120654",
    "jasmine": "registry:dt/jasmine#2.2.0+20160412134438",
    "node": "registry:dt/node#4.0.0+20160509154515"
  }
}
\end{verbatim}

\subsection{: systemjs.config.js}
\label{sec-1-4}
\begin{verbatim}
systemjs.config.js
\end{verbatim}
\begin{verbatim}
/**
 * System configuration for Angular 2 samples
 * Adjust as necessary for your application needs.
 */
(function(global) {
  // map tells the System loader where to look for things
  var map = {
    'app':                        'app', // 'dist',
    '@angular':                   'node_modules/@angular',
    'angular2-in-memory-web-api': 'node_modules/angular2-in-memory-web-api',
    'rxjs':                       'node_modules/rxjs'
  };
  // packages tells the System loader how to load when no filename and/or no extension
  var packages = {
    'app':                        { main: 'main.js',  defaultExtension: 'js' },
    'rxjs':                       { defaultExtension: 'js' },
    'angular2-in-memory-web-api': { main: 'index.js', defaultExtension: 'js' }
  };
  var ngPackageNames = [
    'common',
    'compiler',
    'core',
    'http',
    'platform-browser',
    'platform-browser-dynamic',
    'router',
    'router-deprecated',
    'upgrade',
  ];
  // Add package entries for angular packages
  ngPackageNames.forEach(function(pkgName) {
    packages['@angular/'+pkgName] = { main: pkgName + '.umd.js', defaultExtension: 'js' };
  });
  var config = {
    map: map,
    packages: packages
  }
  System.config(config);
})(this);
\end{verbatim}

\section{App}
\label{sec-2}

NAV => \texttt{Index} / \ref{sec-5} / \ref{sec-4} / \ref{sec-3} / \ref{sec-2-4}

\subsection{: app/main.ts}
\label{sec-2-1}
\begin{verbatim}
app/main.ts
\end{verbatim}
\begin{verbatim}
// TEMP: Imports for loading & configuring the in-memory web api
import { provide } from '@angular/core';
import { XHRBackend } from '@angular/http';

import { InMemoryBackendService, SEED_DATA } from 'angular2-in-memory-web-api';
import { InMemoryDataService } from './in-memory-data.service';

import { bootstrap }    from '@angular/platform-browser-dynamic';
import { HTTP_PROVIDERS } from '@angular/http';

import { AppComponent } from './app.component';

bootstrap(AppComponent, [
    HTTP_PROVIDERS,
    provide(XHRBackend, {
        useClass: InMemoryBackendService }), // in-mem server
    provide(SEED_DATA, {
        useClass: InMemoryDataService }) // in-mem server data
]);
\end{verbatim}

\subsection{: app/app.component.ts}
\label{sec-2-2}
\begin{verbatim}
app/app.component.ts
\end{verbatim}
\begin{verbatim}
import { Component } from '@angular/core';
import { RouteConfig, ROUTER_DIRECTIVES, ROUTER_PROVIDERS } from '@angular/router-deprecated';

import { DashboardComponent } from './dashboard.component';
import { PagesComponent } from './pages.component';
import { PageDetailComponent } from './page-detail.component';
import { PageService } from './page.service';

@Component({
    selector: 'my-app',

    template: `
        <h1>{{title}}</h1>
        <nav>

          <a [routerLink]="['Dashboard']">Dashboard</a>
          <a [routerLink]="['Pages']">Pages</a>
        </nav>
        <router-outlet></router-outlet>
        `,
    styleUrls: ['app/app.component.css'],
    directives: [ROUTER_DIRECTIVES],
    providers: [
        ROUTER_PROVIDERS,
        PageService
    ]
})

@RouteConfig([
    {
        path: '/dashboard',
        name: 'Dashboard',
        component: DashboardComponent,
        useAsDefault: true
    },
    {
        path: '/detail/:id',
        name: 'PageDetail',
        component: PageDetailComponent
    },
    {
        path: '/pages',
        name: 'Pages',
        component: PagesComponent
    }
])

export class AppComponent {
    title = 'Prairie Hill Learning Center';
}
\end{verbatim}

\begin{verbatim}
<a [routerLink]="['Home']">Home</a>
<a [routerLink]="['About']">About</a>
<a [routerLink]="['Programs']">Programs</a>
<a [routerLink]="['Tours']">Tours</a>
<a [routerLink]="['Staff']">Staff</a>
<a [routerLink]="['Calendar']">Calendar</a>
<a [routerLink]="['Employment']">Employment</a>
<a [routerLink]="['Donate']">Donate</a>
<a [routerLink]="['Contact']">Contact</a>
<a [routerLink]="['Events']">Events</a>
\end{verbatim}

\subsection{: app/in-memory-data.service.ts}
\label{sec-2-3}
\begin{verbatim}
app/in-memory-data.service.ts
\end{verbatim}
\begin{verbatim}
export class InMemoryDataService {
  createDb() {
    let pages = [
     { "id": 1,  "title": "Home"       },
     { "id": 2,  "title": "About"      },
     { "id": 3,  "title": "Programs"   },
     { "id": 4,  "title": "Tours"      },
     { "id": 5,  "title": "Staff"      },
     { "id": 6,  "title": "Calendar"   },
     { "id": 7,  "title": "Employment" },
     { "id": 8,  "title": "Donate"     },
     { "id": 9,  "title": "Contact"    },
     { "id": 10, "title": "Events"     }
    ];
    return {pages};
  }
}
\end{verbatim}

\subsection{Dashboard}
\label{sec-2-4}

NAV => \ref{app.component.ts} / \ref{dashboard.component.html} / \ref{page-detail.component.css}
=> \ref{app.component.css} / \ref{dashboard.component.css} 

\subsubsection{: app/dashboard.component.ts}
\label{sec-2-4-1}
\begin{verbatim}
app/dashboard.component.ts
\end{verbatim}
\begin{verbatim}
import { Component, OnInit } from '@angular/core';

import { Page } from './page';
import { PageService } from './page.service';

import { Router } from '@angular/router-deprecated';

@Component({
    selector: 'my-dashboard',
    templateUrl: 'app/dashboard.component.html',
    styleUrls: ['app/dashboard.component.css']
})

export class DashboardComponent implements OnInit {

    pages: Page[] = [];

    constructor(
        private router: Router,
        private pageService: PageService) {
    }

    ngOnInit() {
        this.pageService.getPages()
            .then(pages => this.pages = pages.slice(1,5));
    }

    gotoDetail(page: Page){
        let link = ['PageDetail', { id: page.id }];
        this.router.navigate(link);
    }
}
\end{verbatim}

\subsection{Pages}
\label{sec-2-5}

TEMP => \ref{dashboard.component.html} / \ref{page-detail.component.html} / \ref{pages.component.html}
SERV => \ref{page.service.ts} 
COMP => \ref{page-detail.component.ts} / \ref{pages.component.ts} / \ref{dashboard.component.ts}
STYL => \ref{pages.component.css} /

\subsubsection{: app/page.ts}
\label{sec-2-5-1}
\begin{verbatim}
app/page.ts
\end{verbatim}
\begin{verbatim}
export class Page {
    id: number;
    title: string;
}
\end{verbatim}

\subsubsection{: app/pages.component.ts}
\label{sec-2-5-2}
\begin{verbatim}
app/pages.component.ts
\end{verbatim}
\begin{verbatim}
import { Component, OnInit } from '@angular/core';
import { Router } from '@angular/router-deprecated';

import { Page } from './page';
import { PageService } from './page.service';
import { PageDetailComponent } from './page-detail.component';

@Component({
    selector: 'my-pages',
    templateUrl: 'app/pages.component.html',
    styleUrls: ['app/pages.component.css'],
    directives: [PageDetailComponent]
})

export class PagesComponent implements OnInit {
    pages: Page[]; 
    selectedPage: Page;
    addingPage = false;
    error: any;

    constructor(
        private _router: Router,
        private _pageService: PageService) { }

    getPages() {
        this._pageService
            .getPages()
            .then(pages => this.pages = pages)
            .catch(error => this.error = error); //TODO: Display error message
    }

    addPage() {
        this.addingPage = true;
        this.selectedPage = null;
    }

    close(savedPage: Page) {
        this.addingPage = false;
        if (savedPage) { this.getPages(); }
    }

    delete(page: Page, event: any) {
        event.stopPropagation();
        this._pageService
            .delete(page)
            .then(res => {
                this.pages = this.pages.filter(h => h !== page);
                if (this.selectedPage === page) {
                    this.selectedPage = null; }
            })
            .catch(error => this.error = error); // TODO: Display error message
    }

    ngOnInit() {
        this.getPages();
    }

    onSelect(page: Page) {
        this.selectedPage = page;
        this.addingPage = false;
    }

    gotoDetail() {
        this._router.navigate(['PageDetail', {
            id: this.selectedPage.id }]);
    }
}
\end{verbatim}

\subsubsection{: app/page-detail.component.ts}
\label{sec-2-5-3}
\begin{verbatim}
app/page-detail.component.ts
\end{verbatim}
\begin{verbatim}
import { Component, Input, Output, OnInit, EventEmitter } from '@angular/core';
import { RouteParams } from '@angular/router-deprecated';

import { Page } from './page';
import { PageService } from './page.service';

@Component({
    selector: 'my-page-detail',
    templateUrl: 'app/page-detail.component.html',
    styleUrls: ['app/page-detail.component.css']
})

export class PageDetailComponent implements OnInit {
    @Input() page: Page;
    @Output() close = new EventEmitter();
    error: any;
    navigated = false; // true if navigated here

    constructor(
        private _pageService: PageService,
        private _routeParams: RouteParams) {
    }

    ngOnInit() {
        if (this._routeParams.get('id') !== null) {
            let id = +this._routeParams.get('id');
            this.navigated = true;
            this._pageService.getPage(id)
                .then(page => this.page = page);
        } else {
            this.navigated = false;
            this.page = new Page();
        }
    }

    save() {
        this._pageService
            .save(this.page)
            .then(page => {
                this.page = page; // saved page, w/ id if new
                this.goBack(page);
            })
            .catch(error => this.error = error); // TODO: Display error message
    }

    goBack(savedPage: Page = null) {
        this.close.emit(savedPage);
        if (this.navigated) {
            window.history.back();
        }
    }
}
\end{verbatim}

\subsubsection{: app/page.service.ts}
\label{sec-2-5-4}
\begin{verbatim}
app/page.service.ts
\end{verbatim}
\begin{verbatim}
import { Injectable } from '@angular/core';
import { Http, Headers } from '@angular/http';

import 'rxjs/add/operator/toPromise';

import { Page } from './page';

@Injectable()
export class PageService {

    private pagesUrl = 'app/pages'; // URL to web api

    constructor(private http: Http) { }

    // CREATE new Page
    private post(page: Page): Promise<Page> {
        let headers = new Headers({
            'Content-Type': 'application/json'});

        return this.http
            .post(this.pagesUrl,
                  JSON.stringify(page),
                  {headers: headers})
            .toPromise()
            .then(res => res.json().data)
            .catch(this.handleError);
    }

    // READ existing Page(s)
    getPages(): Promise<Page[]> {
        return this.http.get(this.pagesUrl)
            .toPromise()
            .then(response => response.json().data)
            .catch(this.handleError);
    }
    getPage(id: number) {
        return this.getPages()
            .then(pages => pages.filter(page => page.id === id)[0]);
    }

    // UPDATE existing Page
    private put(page: Page) {
        let headers = new Headers();
        headers.append('Content-Type',
                       'application/json');

        let url = `${this.pagesUrl}/${page.id}`;

        return this.http
            .put(url, JSON.stringify(page),
                 {headers: headers})
            .toPromise()
            .then(() => page)
            .catch(this.handleError);
    }

    // DESTROY existing Page
    delete(page: Page) {
        let headers = new Headers();
        headers.append('Content-Type',
                       'application/json');

        let url = `${this.pagesUrl}/${page.id}`;

        return this.http
            .delete(url, headers)
            .toPromise()
            .catch(this.handleError);
    }

    // SAVE combination of _post and _put methods
    save(page: Page): Promise<Page> {
        if (page.id) {
            return this.put(page);
        }
        return this.post(page);
    }

    // handle ERRORS
    private handleError(error: any) {
        console.error('An error occurred', error);
        return Promise.reject(error.message || error);
    }
}
\end{verbatim}

\section{Template}
\label{sec-3}

APP  => \ref{sec-2} /
SERV => \ref{page.service.ts} /
COMP => \ref{page-detail.component.ts} / \ref{pages.component.ts}
TEMP => \ref{page-detail.component.html}

\subsection{: index.html}
\label{sec-3-1}
\begin{verbatim}
index.html
\end{verbatim}
\begin{verbatim}
<html>
  <head>
    <base href="/">

    <title>Prairie Hill Learning Center</title>
    <meta charset="UTF-8">
    <meta name="viewport" content="width=device-width, initial-scale=1">
    <!--<link rel="stylesheet" href="css/pure-release-0.6.0/pure-min.css">-->
    <link rel="stylesheet" href="styles.css">
    <link href='//fonts.googleapis.com/css?family=Lobster|Roboto:400,100,100italic,700italic,700|Clicker+Script|Kaushan+Script|News+Cycle:400,700|BenchNine|Poiret+One|Open+Sans+Condensed:300|Playball|Shadows+Into+Light+Two' rel='stylesheet' type='text/css'>

    <!-- 1. Load libraries -->
     <!-- Polyfill(s) for older browsers -->
    <script src="node_modules/core-js/client/shim.min.js"></script>
    <script src="node_modules/zone.js/dist/zone.js"></script>
    <script src="node_modules/reflect-metadata/Reflect.js"></script>
    <script src="node_modules/systemjs/dist/system.src.js"></script>

    <!--<script src="https://www.gstatic.com/firebasejs/3.0.0/firebase.js"></script>-->

    <!-- 2. Configure SystemJS -->
    <script src="systemjs.config.js"></script>
    <script>
     System.import('app').catch(function(err){ console.error(err); });
    </script>
  </head>
  <!-- 3. Display the application -->
  <body>
    <div id="header" styleName="pure-g">
      <phill-header>...</phill-header>
    </div>
    <div id="main" styleName="pure-g">
      <my-app>Loading...</my-app>
    </div>
    <div id="footer" styleName="pure-g">
      <phill-footer>...</phill-footer>
    </div>
  </body>
</html>
\end{verbatim}

\subsection{Dashboard}
\label{sec-3-2}

NAV => \ref{dashboard.component.ts} / \ref{pages.component.ts} /

\subsubsection{: app/dashboard.component.html}
\label{sec-3-2-1}
\begin{verbatim}
app/dashboard.component.html
\end{verbatim}
\begin{verbatim}
<h3>Prairie Hill Pages (Spaces)</h3>
<div class="grid grid-pad">
  <div *ngFor="let page of pages"
       (click)="gotoDetail(page)" class="col-1-4">
    <div class="module page">
      <h4>{{page.title}}</h4>
    </div>
  </div>
</div>
\end{verbatim}

\subsection{Pages}
\label{sec-3-3}

\subsubsection{: app/pages.component.html}
\label{sec-3-3-1}
\begin{verbatim}
app/pages.component.html
\end{verbatim}
\begin{verbatim}
<h2>My Pages</h2>
<ul class="pages">
  <li *ngFor="let page of pages"
      (click)="onSelect(page)"
      [class.selected]="page === selectedPage">
    <span class="page-element">
      <span class="badge">{{page.id}}</span> {{page.title}}
    </span>
    <button class="delete-button"
            (click)="delete(page, $event)">Delete</button>
  </li>
</ul>

<button (click)="addPage()">Add New Page</button>
<div *ngIf="addingPage">
  <my-page-detail (close)="close($event)"></my-page-detail>
</div>

<div *ngIf="selectedPage">
  <h2>
    {{selectedPage.title | uppercase}} is your current page
  </h2>
  <button (click)="gotoDetail()">View Details</button>
</div>
\end{verbatim}

\subsubsection{: app/page-detail.component.html}
\label{sec-3-3-2}
\begin{verbatim}
app/page-detail.component.html
\end{verbatim}
\begin{verbatim}
<div *ngIf="page">
  <h2>{{page.title}}</h2>
  <div>
    <label>id: </label>{{page.id}}
  </div>
  <div>
    <label>title: </label>
    <input [(ngModel)]="page.title" placeholder="title"/>
  </div>
  <button (click)="goBack()">Back</button>
  <button (click)="save()">Save</button>
</div>
\end{verbatim}

\section{Styles}
\label{sec-4}

\ref{sec-5} / \ref{sec-2}

\subsection{: styles.css}
\label{sec-4-1}
\begin{verbatim}
styles.css
\end{verbatim}
\begin{verbatim}
h1 {
  color: #369;
  font-family: Arial, Helvetica, sans-serif;
  font-size: 250%;
}
h2 { 
  color: #444;
  font-family: Arial, Helvetica, sans-serif;   
  font-weight: lighter;
}
body { 
  margin: 2em; 
}
body, input[text], button { 
  color: #888; 
  font-family: Cambria, Georgia; 
}
button {
  font-family: Arial;
  background-color: #eee;
  border: none;
  padding: 5px 10px;
  border-radius: 4px;
  cursor: pointer;
  cursor: hand;
}
button:hover {
  background-color: #cfd8dc;
}
button:disabled {
  background-color: #eee;
  color: #aaa; 
  cursor: auto;
}
/* everywhere else */
* { 
  font-family: Arial, Helvetica, sans-serif; 
}
\end{verbatim}

NAV => \ref{index.html} / \ref{pages.component.html} / \ref{pages.component.ts}

\subsection{: app/app.component.css}
\label{sec-4-2}
\begin{verbatim}
app/app.component.css
\end{verbatim}
\begin{verbatim}
h1 {
  font-size: 1.2em;
  color: #999;
  margin-bottom: 0;
}
h2 {
  font-size: 2em;
  margin-top: 0;
  padding-top: 0;
}
nav a {
  padding: 5px 10px;
  text-decoration: none;
  margin-top: 10px;
  display: inline-block;
  background-color: #eee;
  border-radius: 4px;
}
nav a:visited, a:link {
  color: #607D8B;
}
nav a:hover {
  color: #039be5;
  background-color: #CFD8DC;
}
nav a.router-link-active {
  color: #039be5;
}
\end{verbatim}

\subsection{Dashboard}
\label{sec-4-3}

\subsubsection{: app/dashboard.component.css}
\label{sec-4-3-1}
\begin{verbatim}
app/dashboard.component.css
\end{verbatim}
\begin{verbatim}
[class*='col-'] {
  float: left;
}
*, *:after, *:before {
    -webkit-box-sizing: border-box;
    -moz-box-sizing: border-box;
    box-sizing: border-box;
}
h3 {
  text-align: center; margin-bottom: 0;
}
[class*='col-'] {
  padding-right: 20px;
  padding-bottom: 20px;
}
[class*='col-']:last-of-type {
  padding-right: 0;
}
.grid {
  margin: 0;
}
.col-1-4 {
  width: 25%;
}
.module {
    padding: 20px;
    text-align: center;
    color: #eee;
    max-height: 120px;
    min-width: 120px;
    background-color: #607D8B;
    border-radius: 2px;
}
h4 {
  position: relative;
}
.module:hover {
  background-color: #EEE;
  cursor: pointer;
  color: #607d8b;
}
.grid-pad {
  padding: 10px 0;
}
.grid-pad > [class*='col-']:last-of-type {
  padding-right: 20px;
}
@media (max-width: 600px) {
    .module {
      font-size: 10px;
      max-height: 75px; }
}
@media (max-width: 1024px) {
    .grid {
      margin: 0;
    }
    .module {
      min-width: 60px;
    }
}
\end{verbatim}

\subsection{Pages}
\label{sec-4-4}

\subsubsection{: app/pages.component.css}
\label{sec-4-4-1}
\begin{verbatim}
app/pages.component.css
\end{verbatim}
\begin{verbatim}
.selected {
    background-color: #CFD8DC !important;
    color: white;
}
.pages {
    margin: 0 0 2em 0;
    list-style-type: none;
    padding: 0;
    width: 15em;
}
.pages li {
    cursor: pointer;
    position: relative;
    left: 0;
    background-color: #EEE;
    margin: .5em;
    padding: .3em 0;
    height: 1.6em;
    border-radius: 4px;
}
.pages li.selected:hover {
    background-color: #BBD8DC !important;
    color: white;
}
.pages li:hover {
    color: #607D8B;
    background-color: #DDD;
    left: .1em;
}
.pages .text {
    position: relative;
    top: -3px;
}
.pages .badge {
    display: inline-block;
    font-size: small;
    color: white;
    padding: 0.8em 0.7em 0 0.7em;
    background-color: #607D8B;
    line-height: 1em;
    position: relative;
    left: -1px;
    top: -4px;
    height: 1.8em;
    margin-right: .8em;
    border-radius: 4px 0 0 4px;
}
\end{verbatim}

\subsubsection{: app/page-detail.component.css}
\label{sec-4-4-2}
\begin{verbatim}
app/page-detail.component.css
\end{verbatim}
\begin{verbatim}
label {
  display: inline-block;
  width: 3em;
  margin: .5em 0;
  color: #607D8B;
  font-weight: bold;
}
input {
  height: 2em;
  font-size: 1em;
  padding-left: .4em;
}
button {
  margin-top: 20px;
  font-family: Arial;
  background-color: #eee;
  border: none;
  padding: 5px 10px;
  border-radius: 4px;
  cursor: pointer; cursor: hand;
}
button:hover {
  background-color: #cfd8dc;
}
button:disabled {
  background-color: #eee;
  color: #ccc; 
  cursor: auto;
}
\end{verbatim}

\section{Dev}
\label{sec-5}

\subsection{Mon May 23 21:42:18 CDT 2016}
\label{sec-5-1}



\subsection{Sun May 22 14:10:19 CDT 2016}
\label{sec-5-2}

\url{https://angular.io/docs/ts/latest/tutorial/toh-pt5.html}

Routing

\begin{itemize}
\item $\square$ turn \ref{sec-2} into an application shell that only handles navigation
\item $\square$ relocate \emph{Pages} concerns within the current \texttt{app.component.js} to a separate
\texttt{PagesComponent}
\item $\square$ add routing
\item $\square$ create a new \texttt{DashboardComponent}
\item $\square$ tie the \emph{Dashboard} into the navigation structure
\end{itemize}

\subsection{Sat May 21 22:28:33 CDT 2016}
\label{sec-5-3}

\url{https://angular.io/docs/js/latest/quickstart.html}

Angular2 is written with TypeScript(ES6). This is the future.

\url{https://angular.io/docs/ts/latest/quickstart.html}

\begin{enumerate}
\item Create the app's project folder and define package dependencies and special
project setup

\begin{enumerate}
\item Create the project folder

You are in \href{./}{it}.

\item Add package definitiion and configuration files

\ref{sec-1}

\ref{package.json}
\ref{tsconfig.json}
\ref{typings.json}
\ref{systemjs.config.js}

\item Install packages

\begin{verbatim}
npm install
\end{verbatim}

\begin{itemize}
\item $\boxminus$ npm WARN
\begin{itemize}
\item $\boxtimes$ optional

\begin{verbatim}
Skipping failed optional dependency /chokidar/fsevents:
\end{verbatim}

\url{https://github.com/paulmillr/chokidar/issues/425}

"It's just a warning, not an error. You can safely ignore it. 
Fsevents is an optional dependency and is used on only on OSX."
--nono

\item $\boxtimes$ notsup

\begin{verbatim}
Not compatible with your operating system or architecture: fsevents@1.0.12
\end{verbatim}

\item $\square$ phill-angular2@1.0.0 No repository field.
\end{itemize}
\end{itemize}
\end{enumerate}

\item Create the app's Angular root component

\rule{\linewidth}{0.5pt}

\ref{sec-1} / \ref{sec-2}

\rule{\linewidth}{0.5pt}
app/\ref{app.component.ts}

Structure of every component:

\rule{\linewidth}{0.5pt}

\begin{itemize}
\item One or more \texttt{import} statments to reference the things we need.
\item A \texttt{@Component decorator} that tells Angular what template to use and how to
create the component.
\item A \texttt{component class} that controls the appearance and behavior of a view 
through its template.
\end{itemize}

\item Add \ref{main.ts}, identifying the root component to Angular

\ref{sec-2}

app/\ref{main.ts}

\item Add \ref{index.html}, the web page that hosts the application

\ref{sec-3}

\item Build and run the app

\begin{verbatim}
npm start
\end{verbatim}
\end{enumerate}

\subsection{Tutorials}
\label{sec-5-4}

\url{https://angular.io/docs/ts/latest/tutorial/toh-pt3.html}

\ref{app.component.ts}

\ref{page-detail.component.ts}

\ref{sec-2}
% Emacs 24.5.1 (Org mode 8.2.10)
\end{document}
